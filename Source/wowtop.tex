\documentclass[
%paper=5.5in:8.5in,
a5paper,
]{scrbook} % Document font size and paper size


\usepackage{bindery}
\usepackage{dropcaps}
\usepackage{illustrations}
\usepackage{artheadings}



\makeatletter
\@ifundefined{ifpictures}{\newif\ifpictures}{}
\makeatother
\picturestrue
%\picturesfalse

\ifpictures
    \includecomment{pictures}
    \excludecomment{placeholder}
\else
    \excludecomment{pictures}
    \includecomment{placeholder}
\fi

\setdropcaps{Mosaic_i.ttf}
\newfontfamily\mytitlefont{AMERIKA.ttf}
\renewfontfamily\chapterfont{AMERIKA.ttf}

\renewcommand{\partname}{Book}
\renewcommand*{\chapterformat}{\chaptername~\thechapter}
\renewcommand*{\partformat}{\partname~\thepart}


\BeforeTOCHead[toc]{%
\DeclareTOCStyleEntry[dynnumwidth=true,beforeskip=0.2cm,linefill=\TOCLineLeaderFill]{chapter}{chapter}
\DeclareTOCStyleEntry[beforeskip=.3cm,dynnumwidth=true,entryformat=\Large\bfseries]{part}{part}
}


\DeclareTOCStyleEntry[
    beforeskip=.1cm,
  %  indent=20pt,  % Adjust this value to match your other entries
 %   numwidth=0pt,
    entrynumberformat=\hideentrynumber,
    linefill=\TOCLineLeaderFill
]{tocline}{figure}



\renewcommand{\thechapter}{\Roman{chapter}}

\automark{chapter}
\lehead{The War of the Worlds}
\renewcommand{\chaptermark}[1]{\markboth{\chaptername\ \arabic{chapter}:\ #1}{}}
%we're using Roman numerals in the chapter names themselves and it's a PITA to decode Roman numerals each time: redefine chaptermark to use Arabic numerals in the header
\rohead{\leftmark}
\flushbottom

\hyphenation{}

%\DeclareEmphSequence{\itshape,\upshape}
\begin{document}


\frontmatter
\includepdf[width=1.3\textwidth]{titlepage.jpg}


\pagestyle{plain}

\KOMAoptions{headings=openleft}

\tableofcontents

%\renewcommand{\listfigurename}{List of Illustrations}
%\listoffigures


\mainmatter

\renewcommand*{\chapterpagestyle}{plain}
                    
 \KOMAoptions{headings=openany}
\begin{center}
\begin{quote}
But who shall dwell in these worlds if they be inhabited?\textellipsis Are we or they Lords of the World?\textellipsis And how are all things made for man?\\

\textsc{Kepler} (quoted in \textit{The Anatomy of Melancholy})

\end{quote}
\end{center}
\thispagestyle{empty}
\clearpage

\pagestyle{headings}

\KOMAoptions{headings=openleft}
\ArtPart[The Coming of the Martians]{part1}
\KOMAoptions{headings=openright}

\include{chapters/01.tex}
\include{chapters/02.tex}
\include{chapters/03.tex}
\include{chapters/04.tex}
\include{chapters/05.tex}
\include{chapters/06.tex}
\include{chapters/07.tex}
\include{chapters/08.tex}
\include{chapters/09.tex}
\include{chapters/10.tex}
\include{chapters/11.tex}
\include{chapters/12.tex}
\include{chapters/13.tex}
\include{chapters/14.tex}
\include{chapters/15.tex}
\include{chapters/16.tex}
\include{chapters/17.tex}

\addtocontents{toc}{\protect\newpage}
\KOMAoptions{headings=openleft}
\ArtPart[The Earth Under the Martians]{part2}
\KOMAoptions{headings=openright}

\include{chapters/18.tex}
\include{chapters/19.tex}
\include{chapters/20.tex}
\include{chapters/21.tex}
\include{chapters/22.tex}
\include{chapters/23.tex}
\include{chapters/24.tex}
\include{chapters/25.tex}
\include{chapters/26.tex}
\include{chapters/27.tex}

%
%\backmatter
\KOMAoptions{headings=openleft}
%\include{chapters/landingsites.tex}
%\include{chapters/journey.tex}
%\include{chapters/brosjourney1.tex}
%\include{chapters/brosjourney2.tex}

\chapter*{Colophon}
\centering
\vfill
\vfill
\begin{minipage}{\textwidth}
\textit{The War of the Worlds} (Herbert George »H.G.« Wells, 1866\textendash1946) was serialised in \textit{Pearson's Magazine} (in the UK) and \textit{Cosmopolitan} magazine (in the US) in 1897. In 1898, it was published as a standalone work by William Heinemann (in the UK) and Harper \& Bros. (in the US). 
\end{minipage}
\vfill
gutenberg.org/ebooks/36
\vfill
\rule{0.5\textwidth}{.4pt}
\vfill
\begin{minipage}{\textwidth}
Illustrations by Brazilian artist Henrique Alvim Corrêa (1876\textendash1910) are from the first French edition: \textit{La guerre des mondes}, translated by Henry Durand-Davray and published in 1906 by L.~Vandamme \& Co., in Brussels (Belgium).
\end{minipage}
\vfill
\rule{0.5\textwidth}{.4pt}
\vfill
\begin{minipage}{\textwidth}
Text is set in »EB Garamond«, Georg Mayr-Duffner's free and open source implementation of Claude Garamond’s famous humanist typefaces from the mid-sixteenth century.  Chapter dropcaps are set in »Mosaic Initials«, by Paul Lloyd. Title page, chapter headings, and part headings are set in »Amerika«, by Apostrophic Labs.
\end{minipage}
\vfill
github.com/georgd/EB-Garamond\\
moorstation.org/typoasis/designers/lloyd/\\
moorstation.org/typoasis/designers/lab/
%\vfill
%\rule{0.5\textwidth}{.4pt}
%\vfill
%\begin{minipage}{\textwidth}
%%Maps used are »Outer London«, by Edward Stanford, first published in \textit{Stanford's London Atlas of Universal Geography Exhibiting the Physical and Political Divisions of the Various Countries of the World} (1901); and »The Home Counties \& the East of England«, published in the 1908 edition of \textit{The Harmsworth Atlas and Gazetter}.
%\end{minipage}
\vfill
\rule{0.5\textwidth}{.4pt}
\vfill
\begin{minipage}{\textwidth}
This typeset is dedicated to the public domain under a Creative Commons CC0 1.0 Universal deed: creativecommons.org/publicdomain/zero/1.0/
\end{minipage}
\vfill
\rule{0.5\textwidth}{.4pt}
\vfill
Typeset in \LaTeX{}. Last revised \today.

\thispagestyle{empty}
\end{document}